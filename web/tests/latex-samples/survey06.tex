\documentclass[11pt,a4paper]{report}

\usepackage{verbatim}
\usepackage{epsfig}
\usepackage{theorem}
\usepackage{graphicx}
\usepackage{amsmath}
\usepackage{amssymb}
\usepackage[dvips]{color}
\usepackage[dvipdfm,pdftitle={Cryptography and Security - Survey 1},pdfpagemode=None,colorlinks=true,linkcolor=black,urlcolor=blue]{hyperref}
\usepackage{fullpage}
\usepackage{subfigure}
\usepackage{algorithmic}
\usepackage{algorithm}
\usepackage{lib/verbtext}
\usepackage{lib/security}


\input{lib/macros.tex}



% +-[A décommenter pour avoir la solution]--------------+


%\solutiontrue



% +-[document]------------------------------------------+
\begin{document}


% +-[header survey]-------------------------------------+
\input{lib/header_fullpage.tex}
\begin{center}
\sf
  \Large{Cryptography and Security}

  \vspace{0.3cm}
  \Large{2012 -- 2013}

  \vspace{0.3cm}
  \Large{Survey n$^{\text{o}}$ 6}

  \vspace{0.5cm}
\end{center}

\hfill{Name:~}{\vrule width 7cm height 0.4pt}

\bigskip

\hfill{SCIPER n$^{\text{o}}$:~}{\vrule width 7cm height 0.4pt}

\begin{comment}
 \question[]{}
 {}
 {}
 {}
 {}
\end{comment}

\question[2]{Forward secrecy implies that \dots}
{the cipher is perfectly secure.}               
% hello
{even if a long term key is disclosed, the communication remains private.}
{the secrecy of a message depends on the security of the next message.}
{when an adversary forwards an encrypted message to an decryption oracle, the
message remains secret.}

\question[1]{Tick the \emph{false} assumption.}
{Static Diffie-Hellman has forward secrecy.}
{If we run the static Diffie-Hellman protocol between Alice and Bob, the
communications will always be the same.}
{Static Diffie-Hellman can be implemented over elliptic curves.}
{In ephemeral Diffie-Hellman, $g^x$ and $g^y$ are discarded at the end of the
protocol.}
 
\question[2]{Tick the \emph{correct} assertion.}
{The ElGamal public-key is a hash of the secret key.}
{For the ElGamal decryption, we need to compute an exponential.}
{The ElGamal secret key is picked at random in $\mathbb{Z}_n$, with $n = pq$ for
$p,q$ primes. }
{The ElGamal ciphertext is an element of $\mathbb{Z}_n^*$, with $n=pq$ for $p,q$
primes.}

\question[3]{Which of the following encryption schemes is deterministic?}
{RSA-OAEP}
{Plain ElGamal}
{Plain Rabin}
{PKCS\#1.5}


\question[4]{Plain RSA (with an $\ell$-bit modulus) \dots}
{is commonly used in practice.}
{decrypts in $O(\ell^2)$ time.}
{encrypts in $O(\ell)$ time.}
{has homomorphic properties.}

\question[2]{Tick the \emph{correct} assertion.}
{ElGamal encryption has a slower key-generation algorithm than RSA.}
{ElGamal encryption is length increasing.}
%{encrypts a message in $O(\ell^2)$.}
{ElGamal cryptography cannot be used to sign a message.}
{DSS is based on factoring.}

 \question[3]{Tick the \emph{incorrect} assertion.}
 {Plain RSA is deterministic.}
 {An RSA modulus is a product of two prime numbers.}
 {Deterministic encryption schemes always provide forward secrecy.}
 {DSS requires public parameters.}
 
 \question[4]{KEM \dots}                   
 		{stands for Keyless Encryption Mechanism.}
    {is a Korean encryption mechanism.}
 {is a symmetric-key algorithm.}
 {is a public-key algorithm.}
  
%  \question[1]{Tick the \emph{incorrect} assertion. Plain RSA is \textit{not} preferred to use in practice since \dots}
%{key length is too short.}
%{it has homomorphic properties.}
%{messages are not defined properly in the input domain.}
%{ its engineering does not seem secure.}
\question[2]{Tick the \emph{incorrect} assertion.}
{A signature scheme may provide message recovery.}
%{The short key-length of Plain RSA is what makes it not preferred in
%practice.}
      {We don't like to use Plain RSA in practice because of its short key-length.}		
	 	{In the Rabin cryptosystem, ambiguity in the decryption is prevented by adding redundancy in the plaintext.}
{A trapdoor function is easy to compute in one direction, yet believed to be
difficult to compute in the opposite direction without the trapdoor.}


 \question[1]{Tick the \emph{incorrect} assertion.}
 {Commitment schemes never use randomness.}
      {A commitment scheme can be based on the hardness of the discrete logarithm problem.}
 {A commitment scheme should be hiding and binding.}
 {Perdersen Commitment uses two large primes.}



\end{document}




